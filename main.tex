\documentclass[11pt,a4paper,twoside,
openright]{book}

\usepackage{fullpage}
\usepackage{amsmath}
\usepackage{amsthm}
\usepackage{amsfonts}
\usepackage{tikz-cd}
\usepackage{chngcntr}
\usepackage[utf8x]{inputenc}
\usepackage{enumerate}

\newenvironment{dedication}
  {
   \thispagestyle{empty}% no header and footer
   \vspace*{\stretch{1}}% some space at the top
   \itshape             % the text is in italics
   \raggedleft          % flush to the right margin
  }
  {\par % end the paragraph
   \vspace{\stretch{3}} % space at bottom is three times that at the top
   \clearpage           % finish off the page
  }

\begin{document}

\begin{titlepage}
	
	\begin{figure}
		\centering
		\includegraphics[width=424pt]{tesiSCIENZE_TECNOLOGIE.jpg}
		\vspace{0.5 cm}
	\end{figure}
	

\begin{center}
{\LARGE Corso di Laurea in Informatica Magistrale}
\end{center}

\begin{center}
\vspace{3 cm}
{\Large \textsc{Serverless application for multiple datebases governance on AWS} }
\end{center}
\par
  \vspace{3 cm}
  
  \begin{flushleft}
  		 Relatore:\\ Elena PAGANI
		 
  		 \noindent Correlatore:\\ Roy S. HALSTEAD
  \end{flushleft}
  \vspace{1 cm}
  \begin{flushright}
  	Tesi di Laurea di:\\ Michael DANIEL NAGUIB\\ Matricola: 923425
  \end{flushright}
    	  
\vfill
\begin{center}
	{\large Anno Accademico 20XX/20XX}
\end{center}
\end{titlepage}


\begin{dedication}
test Dedication
\end{dedication}

\tableofcontents 

\chapter{Introduction}
Abstract goes here

% 1.  Introduzione
% Descrizione del problema da risolvere e degli obiettivi da raggiungere.  Se esiste, descrizione della situazione di partenza e degli aspetti negativi per cui si è deciso di intervenire.
%prendere su teams di roy: leggere '[477029] Allegato 2 - Requisiti'
% Evitiamo la pubblicità dell'azienda.


\chapter{Reference Technologies}
\section{AGILE Software Development}
\section{Cloud Infrastructure}
\subsection{Serverless Architecture}
\section{Big Data Platform}
\section{ToolBox}
% 2.  Tecnologie di riferimento
% 2.1  Programmazione agile  (standup mattutino, usavamo trello)
%infrastructure-as-a-code -> cloud formation 
% 2.2  Infrastrutture cloud  (caratteristiche in particolare di quella considerata)
% 2.2.1  Architetture serverless
% 2.3  Piattaforme Big Data  (caratteristiche e requisiti) chiedere?
% 2.4  Strumenti utilizzati  (es. React, container e microservizi, , nodejs, k8s ...)
% In questo capitolo non deve scrivere tutto lo scibile riguardo a questi argomenti, ma solo quanto serve per far capire cosa lei ha fatto, dando riferimenti bibliografici per ciò che non è rilevante ai fini del suo lavoro.


\chapter{The Platform}
\section{Requirements}
\section{Architecture}
\section{Implementation}
% 3.  Descrizione della piattaforma
% 3.1  Requisiti  (es. tipologie di utenti, funzionalità quali aggiunta DB/gestione privilegi/gestione audit...)
% 3.2  Architettura  (evidenziando quali moduli funzionali ha sviluppato lei, e i loro servizi)
% 3.3  Aspetti implementativi  (qui non deve commentare tutto il codice, ma solo descrivere gli aspetti più importanti o problematici del lavoro, evidenziando come ha affrontato i problemi o che idee originali ha avuto nello sviluppo.  Poche righe di codice se e solo se proprio le servono per spiegare qualcosa che ha fatto.)
    %-> knex con i multi db, usato perchè non serve un orm essendo dinamico
    %-> gestione dei layer con ottimizzazione del caricamento dei layer sulle lambda
    %-> con cloud formation la risoluzione del nome di apiGatewayDeployment
    %-> FE parte user che è dinamica (es metadati per la generazione della pagina dinamica, catalogazione dei tipi custom)


\chapter{Validation \& Performance}
% 4.  Validazione e prestazioni (validazione ?)
% * metodologia seguite (?)
% * risultati ottenuti rispetto agli obiettivi iniziali
    %-> ora il cliente può gestire più db in unica platform
    %-> prima dovevano fare uno sviluppo di una pagina per ogni nuova tabella da gestire

\chapter{Conclusions}
% 5.  Conclusioni e sviluppi futuri
% -> si può citare deloitte?
% -> la platform si può usare ovunque ora 
% sviluppi futuri 
    % integrare un processo di approvazione: prima di modificare il dato, la modifica del dato deve rimanere pending fino all'approvazione di un admin/approver
    % integrazione di elementi di analisi dei dati con visualizzazione grafiche di essi

\appendix
\chapter{Bibliography}
% Bibliografia





\end{document}